\documentclass{article}
\usepackage[T1]{fontenc}
\usepackage[utf8]{inputenc}
\usepackage[a4paper]{geometry}
\usepackage{titling}
\usepackage[ngerman]{babel}
\usepackage{hyperref}

\usepackage{minted}
\usemintedstyle{emacs}

\setlength{\droptitle}{-10em}
\renewcommand{\familydefault}{\sfdefault} 

\title{Linux Command Sheet}
\author{Florian Weingartshofer}
\date{\today}

\begin{document}
\maketitle
\tableofcontents{}
\newpage
\section{Basics}
\subsection{Commands}
Ein Command besteht aus verschiedenen Komponenten:

\begin{minted}{cmdlexer.py:CommandLexer -x}
ls --all -l /home/
\end{minted}

\mint{cmdlexer.py:CommandLexer -x}{ls} Das Command selbst

\mint{cmdlexer.py:CommandLexer -x}{--all}
Eine ausgeschriebene Option, diese beginnen mit einem doppeltem Minus.

\mint{cmdlexer.py:CommandLexer -x}{-l}
Die Kurzform einer Option, wird nur mit einem Minus geschrieben.

\mint{cmdlexer.py:CommandLexer -x}{/home/}
Ein Argument, manche Commands können mehrere Argumente erhalten.

\subsection{Files und Directories}
\mint{bash}{file.docx}
Ein normales File.

\mint{bash}{stuff/}
Ein normales Directory, diese enden immer auf ein \mintinline{bash}{/}.

\mint{bash}{.hidden.txt}
Ein verstecktes File, sie beginnen immer mit einem Punkt.

\mint{bash}{.hidden_dir/}
Ein verstecktes Directory beginnt auch mit einem Punkt.

\subsection{Files auflisten}
\mint{cmdlexer.py:CommandLexer -x}{ls}
Zeigt Files und Directories im momentanen Ordner an.

\mint{cmdlexer.py:CommandLexer -x}{ls -l}
Zeigt Files, etc in einer Listenstruktur an.

\mint{cmdlexer.py:CommandLexer -x}{ls -a}
Zeigt \textbf{alle} Files und Ordner an, auch versteckte.

\mint{cmdlexer.py:CommandLexer -x}{ls -la}
Es können auch verschiedene Optionen kombiniert werden.

\mint{cmdlexer.py:CommandLexer -x}{ls -h}

\subsection{Erstellen von Directories}

\mint{cmdlexer.py:CommandLexer -x}{mkdir <directory>}
Ein Directory im momentanen Directory erstellen.

\mint{cmdlexer.py:CommandLexer -x}{mkdir /path/to/dir/newdir/}
Es kann auch ein Directory mit einem Pfad angegeben werden.

Dabei ist das letzte Directory, dass das nur erstellt wird.

\mint{cmdlexer.py:CommandLexer -x}{mkdir -p new/dir}
Mit der P-Flag können auch SUbdirectories rekursiv erstellt werden.

\mint{cmdlexer.py:CommandLexer -x}{mkdir {dir1,dir2,dir3}}
Mit den geschwungene Klammern können mehrer Directories gleichzeitig erstellt werden.

\mint{cmdlexer.py:CommandLexer -x}{mkdir -p new/{dir1,dir2,dir3}}
Die Directories können auch rekursiv in einem Parent-Directory erstellt werden.


\section*{Get Help}
\subsection*{Man-Page}
Linux hat eine Bedinungsanleitung eingebaut, die sogenannte Man-Page.
Diese sind nur zu gebrauchen, wenn man das Command schon kennt.
\mint{cmdlexer.py:CommandLexer -x}{man <command>}

\subsection*{\href{https://tldr.sh/}{TL;DR}}
Ein Tool mit Zusammenfassungen und Beispielen für verschieden Commands: \href{https://tldr.sh/}{https://tldr.sh}

\end{document}
