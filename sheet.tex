\documentclass{article}
\usepackage[T1]{fontenc}
\usepackage[utf8]{inputenc}
\usepackage[a4paper]{geometry}
\usepackage{titling}
\usepackage[ngerman]{babel}
\usepackage{graphicx}
\usepackage{hyperref}
\hypersetup{
    colorlinks=true,
    linkcolor=black,
    filecolor=magenta,      
    urlcolor=cyan,
}

\usepackage{lmodern}
\usepackage{minted}
\newcommand{\cmdinline}[1]{\mintinline{cmdlexer.py:CommandLexer -x}{#1}}
\newcommand{\cmdline}[1]{\mint{cmdlexer.py:CommandLexer -x}{#1}}

\usemintedstyle{emacs}

\setlength{\droptitle}{-10em}
\renewcommand{\familydefault}{\sfdefault} 

\title{Linux Command Sheet}
\author{Florian Weingartshofer}
\date{\today}


\begin{document}
\maketitle
\tableofcontents{}
\newpage
\section{Basics}
\subsection{Filesystem}
Das Linux Filesystem beginnt bei Root (der Pfad lautet: \cmdinline{/}).
Von dort beginnt jeder absolute Pfad.\\
Wenn man ein Terminal startet ist man normalerweiße im Home Directory (\cmdinline{/home/<username>/}).

\subsubsection{Home Directory}
Jeder User hat ein Home Directory, dieses enthält Files welche nur für diesen User gelten, zb Configs, Bilder oder Dokumente.
Das Home Directory liegt in \cmdinline{/home/<username>/}
oder unter der Environment Variable \cmdinline{$HOME}.

\subsubsection{Die Tilde}
In den meisten Shells, wie die Bash, führt die Tilde (\cmdinline{~}) direkt ins Home Directory des Users.
Allerdings hat sie noch andere Funktionen.

\cmdline{~<username>}
Home Directory von <username>.

\cmdline{~-}
Das vorherige Working Directory.

\cmdline{~+}
Das momentane Working Directory.
\\\\
Weiter Infos:
\href{https://stackoverflow.com/questions/998626/meaning-of-tilde-in-linux-bash-not-home-directory}{https://stackoverflow.com}

\subsection{Files und Directories}
\cmdline{file.docx}
Ein normales File.

\cmdline{stuff/}
Ein normales Directory, diese enden immer auf ein \cmdinline{/}.

\cmdline{.hidden.txt}
Ein verstecktes File, sie beginnen immer mit einem Punkt.

\cmdline{.hidden_dir/}
Ein verstecktes Directory beginnt auch mit einem Punkt.

\subsection{Wichtige Files und Directories}
Folgende Files und Directories können im \cmdinline{$HOME} Directory gefunden werden.
\subsubsection{.bashrc}
Das \cmdinline{.bashrc} File ist ein Bash-Script, welches beim öffnen eines Terminals ausgeführt wird.
Man kann es nutzen um Environment Variablen, Aliases zu setzen oder andere Einstellungen vorzunehmen.
Es können auch Splashscreens oder Nachrichten ausgegeben werden.

\begin{minted}{cmdlexer.py:CommandLexer -x}
  echo "Das heutige Wetter"
  curl wttr.in # Download the weather
\end{minted}

\subsubsection{.zshrc}
Ein Init Script für die Z-Shell. Hat die selbe Funktionalität wie das \cmdinline{.bashrc}.

\subsubsection{.config/}
Das \cmdinline{.config/} Directory ist ein Ordner im \cmdinline{.config/}
In diesem Directory werden einige Config-Files gespeichert.

\subsection{Aufbau eines Commands}
Ein Command besteht aus verschiedenen Komponenten:

\begin{minted}{cmdlexer.py:CommandLexer -x}
ls --all -l /home/
\end{minted}

\cmdline{ls} Das Command selbst

\cmdline{--all}
Eine ausgeschriebene Option, diese beginnen mit einem doppeltem Minus.

\cmdline{-l}
Die Kurzform einer Option, wird nur mit einem Minus geschrieben.

\cmdline{/home/}
Ein Argument, manche Commands können mehrere Argumente erhalten.

\subsection{sudo}
Um einen Befehl als Admin auszuführen muss der \cmdinline{sudo} Befehl genutzt werden.

\cmdline{sudo <Befehl>}

\noindent
Beispiel:
\cmdline{sudo rm <file>}

\subsection{Navigieren}
Mit \cmdinline{cd} kann im Terminal navigiert werden.

\cmdline{cd path/to/dir/}
In ein Directory wechseln.

\cmdline{cd}
Wechselt ins Home-Directory.

\cmdline{cd ..}
Ins Parent-Directory wechseln.

\cmdline{cd -}
Ins vorherige Directory wechseln.

\subsection{Files auflisten}
\cmdline{pwd}
Zeigt absoluten Pfad zum momentanen Directory an.

\cmdline{ls}
Zeigt Files und Directories im momentanen Ordner an.

\cmdline{ls <directory>}
Zeigt Files in einem Directory an.

\cmdline{ls -l}
Zeigt Files, etc in einer Listenstruktur an.

\cmdline{ls -a}
Zeigt \textbf{alle} Files und Ordner an, auch versteckte.

\cmdline{ls -h}
Zeigt alle von Menschen lesbare Dateien und Directories an.

\cmdline{ls -la}
Es können auch verschiedene Optionen kombiniert werden.

\subsection{Erstellen von Directories}

\cmdline{mkdir <directory>}
Ein Directory im momentanen Directory erstellen.

\cmdline{mkdir /path/to/dir/newdir/}
Es kann auch ein Directory mit einem Pfad angegeben werden.

Dabei ist das letzte Directory, dass das nur erstellt wird.

\cmdline{mkdir -p new/dir}
Mit der p-Flag können auch Subdirectories rekursiv erstellt werden.

\cmdline{mkdir {dir1,dir2,dir3}}
Mit den geschwungene Klammern können mehrer Directories gleichzeitig erstellt werden.

\cmdline{mkdir -p new/{dir1,dir2,dir3}}
Die Directories können auch rekursiv in einem Parent-Directory erstellt werden.

\subsection{Erstellen von Files}

\cmdline{touch </path/to/file.txt>}
Mit \cmdinline{touch} kann ein File erstellt werden.

Falls das File schon exisitiert wird das Datum aktualisiert.

\cmdline{touch {file1,file2,file3}}
So können mehrere Files erstellt werden.

\subsection{Files Löschen}
\cmdline{rm <file>}
Mit \cmdinline{rm} kann ein oder mehrere Files löschen.

\cmdline{rm -r <dir/>}
Mit der r-Flag können auch Directories entfernt werden.

\cmdline{rm -f <file>}
Um ein File zwingend zu löschen wird die f-Flag benötigt.

\subsection{Files verschieben}
\cmdline{mv <arg1> <arg2>}
Mit \cmdinline{mv} kann ein File oder Directory verschoben oder umbenannt werden.

Vergleichbar mit Cut and Paste.

\subsection{Files kopieren}
\cmdline{cp <path/to/file.txt> <path/to/new/file>}

Mit \cmdinline{cp} können Files oder Directories kopiert werden.

\subsection{Alias}
Mit Aliases kann man Commands abkürzen.

\cmdline{alias hello='clear && echo "Hello"'}

\subsection{dos2unix}
\cmdinline{dos2unix} ist kein Standardpaket und muss daher installiert werden.
Es ist ein einfaches Tool, um DOS(Windows) Newlines zu Unix Newlines zu konvertieren. 

\section{Packagemanager}
Im Gegensatz zu Windows braucht man bei Linux keine Software manuell runterzuladen.
Dies wird automatisch mit einem Packagemanager gemacht.

\subsection{apt}
Das Advanced Packaging Tool(\cmdinline{apt}) ist ein Tool um Packages auf Debian, Ubuntu, etc zu installieren.

\subsubsection{Packages installieren}
Um ein Package zu installieren werden meistens Admin Rechte benötigt,
daher muss vor dem Command ein \cmdinline{sudo} geschrieben werden.

\cmdline{apt install <Package_1 Package_2 ... Package_n>}

\subsubsection{Packages suchen}
\cmdline{apt search <Package>}

\section{Get Help}
\subsection{Man-Page}
Linux hat eine Bedinungsanleitung eingebaut, die sogenannte Man-Page.
Diese sind nur zu gebrauchen, wenn man das Command schon kennt.
\cmdline{man <command>}

\subsection{TL;DR}
Ein Tool mit Zusammenfassungen und Beispielen für verschieden Commands: \href{https://tldr.sh/}{https://tldr.sh}

\end{document}
