\documentclass{article}
\usepackage[T1]{fontenc}
\usepackage[utf8]{inputenc}
\usepackage[a4paper]{geometry}
\usepackage{titling}
\usepackage[ngerman]{babel}
\usepackage{hyperref}
\hypersetup{
    colorlinks=true,
    linkcolor=black,
    filecolor=magenta,      
    urlcolor=cyan,
}


\usepackage{lmodern}

\usepackage{minted}
\usemintedstyle{emacs}

\setlength{\droptitle}{-10em}
\renewcommand{\familydefault}{\sfdefault} 

\title{Linux Command Sheet}
\author{Florian Weingartshofer}
\date{\today}

\begin{document}
\maketitle
\tableofcontents{}
\newpage
\section{Basics}

\subsection{Filesystem}
Das Linux Filesystem beginnt bei Root (der Pfad lautet: \mintinline{cmdlexer.py:CommandLexer -x}{/}).
Von dort beginnt jeder absolute Pfad.\\
Wenn man ein Terminal startet ist man normalerweiße im Home Directory (\mintinline{cmdlexer.py:CommandLexer -x}{/home/<username>/}).

\subsubsection{Home Directory}
Jeder User hat ein Home Directory, dieses enthält Files welche nur für diesen User gelten, zb Configs, Bilder oder Dokumente.
Das Home Directory liegt in \mintinline{cmdlexer.py:CommandLexer -x}{/home/<username>/}
oder unter der Environment Variable \mintinline{cmdlexer.py:CommandLexer -x}{$HOME}.

\subsubsection{Die Tilde}
In den meisten Shells, wie die Bash, führt die Tilde (\mintinline{cmdlexer.py:CommandLexer -x}{~}) direkt ins Home Directory des Users.
Allerdings hat sie noch andere Funktionen.

\mint{cmdlexer.py:CommandLexer -x}{~<username>}
Home Directory von <username>.

\mint{cmdlexer.py:CommandLexer -x}{~-}
Das vorherige Working Directory.

\mint{cmdlexer.py:CommandLexer -x}{~+}
Das momentane Working Directory.
\\\\
Weiter Infos:
\href{https://stackoverflow.com/questions/998626/meaning-of-tilde-in-linux-bash-not-home-directory}{https://stackoverflow.com}

\subsection{Files und Directories}
\mint{cmdlexer.py:CommandLexer -x}{file.docx}
Ein normales File.

\mint{cmdlexer.py:CommandLexer -x}{stuff/}
Ein normales Directory, diese enden immer auf ein \mintinline{bash}{/}.

\mint{cmdlexer.py:CommandLexer -x}{.hidden.txt}
Ein verstecktes File, sie beginnen immer mit einem Punkt.

\mint{cmdlexer.py:CommandLexer -x}{.hidden_dir/}
Ein verstecktes Directory beginnt auch mit einem Punkt.

\subsection{Wichtige Files}
\subsubsection{.bashrc}
Das \mintinline{cmdlexer.py:CommandLexer -x}{.bashrc} File ist ein Bash-Script, welches beim öffnen eines Terminals ausgeführt wird.
Man kann es nutzen um Environment Variablen, Aliases zu setzen oder andere Einstellungen vorzunehmen.

\subsection{Commands}
Ein Command besteht aus verschiedenen Komponenten:

\begin{minted}{cmdlexer.py:CommandLexer -x}
ls --all -l /home/
\end{minted}

\mint{cmdlexer.py:CommandLexer -x}{ls} Das Command selbst

\mint{cmdlexer.py:CommandLexer -x}{--all}
Eine ausgeschriebene Option, diese beginnen mit einem doppeltem Minus.

\mint{cmdlexer.py:CommandLexer -x}{-l}
Die Kurzform einer Option, wird nur mit einem Minus geschrieben.

\mint{cmdlexer.py:CommandLexer -x}{/home/}
Ein Argument, manche Commands können mehrere Argumente erhalten.

\subsection{Navigieren}
Mit \mintinline{cmdlexer.py:CommandLexer -x}{cd} kann im Terminal navigiert werden.

\mint{cmdlexer.py:CommandLexer -x}{cd path/to/dir/}
In ein Directory wechseln.

\mint{cmdlexer.py:CommandLexer -x}{cd}
Wechselt ins Home-Directory.

\mint{cmdlexer.py:CommandLexer -x}{cd ..}
Ins Parent-Directory wechseln.

\mint{cmdlexer.py:CommandLexer -x}{cd -}
Ins vorherige Directory wechseln.

\subsection{Files auflisten}
\mint{cmdlexer.py:CommandLexer -x}{pwd}
Zeigt absoluten Pfad zum momentanen Directory an.

\mint{cmdlexer.py:CommandLexer -x}{ls}
Zeigt Files und Directories im momentanen Ordner an.

\mint{cmdlexer.py:CommandLexer -x}{ls <directory>}
Zeigt Files in einem Directory an.

\mint{cmdlexer.py:CommandLexer -x}{ls -l}
Zeigt Files, etc in einer Listenstruktur an.

\mint{cmdlexer.py:CommandLexer -x}{ls -a}
Zeigt \textbf{alle} Files und Ordner an, auch versteckte.

\mint{cmdlexer.py:CommandLexer -x}{ls -h}
Zeigt alle von Menschen lesbare Dateien und Directories an.

\mint{cmdlexer.py:CommandLexer -x}{ls -la}
Es können auch verschiedene Optionen kombiniert werden.

\subsection{Erstellen von Directories}

\mint{cmdlexer.py:CommandLexer -x}{mkdir <directory>}
Ein Directory im momentanen Directory erstellen.

\mint{cmdlexer.py:CommandLexer -x}{mkdir /path/to/dir/newdir/}
Es kann auch ein Directory mit einem Pfad angegeben werden.

Dabei ist das letzte Directory, dass das nur erstellt wird.

\mint{cmdlexer.py:CommandLexer -x}{mkdir -p new/dir}
Mit der p-Flag können auch Subdirectories rekursiv erstellt werden.

\mint{cmdlexer.py:CommandLexer -x}{mkdir {dir1,dir2,dir3}}
Mit den geschwungene Klammern können mehrer Directories gleichzeitig erstellt werden.

\mint{cmdlexer.py:CommandLexer -x}{mkdir -p new/{dir1,dir2,dir3}}
Die Directories können auch rekursiv in einem Parent-Directory erstellt werden.

\subsection{Erstellen von Files}

\mint{cmdlexer.py:CommandLexer -x}{touch </path/to/file.txt>}
Mit \mintinline{cmdlexer.py:CommandLexer -x}{touch} kann ein File erstellt werden.

Falls das File schon exisitiert wird das Datum aktualisiert.

\mint{cmdlexer.py:CommandLexer -x}{touch {file1,file2,file3}} 

\subsection{Alias}
Mit Aliases kann man Commands abkürzen.

\mint{cmdlexer.py:CommandLexer -x}{alias hello='clear && echo "Hello"'}

\subsection{dos2unix}
\mintinline{cmdlexer.py:CommandLexer -x}{dos2unix} ist kein Standardpaket und muss daher installiert werden.
Es ist ein einfaches Tool um DOS(Windows) Newlines zu Unix Newlines zu konvertieren. 

\section{Get Help}
\subsection{Man-Page}
Linux hat eine Bedinungsanleitung eingebaut, die sogenannte Man-Page.
Diese sind nur zu gebrauchen, wenn man das Command schon kennt.
\mint{cmdlexer.py:CommandLexer -x}{man <command>}

\subsection{TL;DR}
Ein Tool mit Zusammenfassungen und Beispielen für verschieden Commands: \href{https://tldr.sh/}{https://tldr.sh}

\end{document}
